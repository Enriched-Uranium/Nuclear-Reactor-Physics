\chapter{核反应堆物理设计和启动物理试验}
\section*{习题}

\begin{exercise}
    在满足设计准则的基础上,\,堆芯物理设计的内容主要有哪些?\,
    \begin{solution}
        \begin{enumerate}[(1)]
            \item 堆芯栅格和装载方案设计;\,
            \item 反应性控制方案设计;\,
            \item 堆芯燃料管理方案设计.\,
        \end{enumerate}
    \end{solution}
\end{exercise}

\begin{exercise}
    什么是核反应堆功率能力?\,
    \begin{solution}
        核反应堆功率能力是在设计阶段或生产阶段,\,核反应堆设计能够提供的最大热输出功率,\,也就是反应堆的理论上限.\,它主要受到反应堆的几何形状、材料属性、燃料特性、系统配置等因素的影响.\,这个数值主要用于计算反应堆的设计和性能评估,\,对于现有的反应堆,\,没有改变反应堆结构和运行条件的情况下,\,其核反应堆功率能力是固定的.\,
    \end{solution}
\end{exercise}

\begin{exercise}
    建立保护梯形、运行梯形的依据是什么?\,思路如何?\,
    \begin{solution}
        \begin{enumerate}[(1)]
            \item 运行梯形图:通过计算足够多的两类工况下的堆芯轴向功率分布,\,可以给出堆芯不同状态下功率峰因子与轴向功率偏移$AO$的对应关系,\,再由这一关系导出\uppercase\expandafter{\romannumeral1}类工况的轴向功率偏差对堆芯功率变化的梯形限制区.\,
            \item 超功率保护梯形图:在\uppercase\expandafter{\romannumeral2}类工况下保证燃料棒的完整性而限制的最大线功率密度,\,同样为轴向功率偏差对堆芯功率变化的梯形限制区.\,
        \end{enumerate}
        运行梯形图和超功率保护梯形图给出了核反应堆在不同功率分布条件下的功率输出范围.\,
    \end{solution}
\end{exercise}

\begin{exercise}
    正常启动到临界过程中主要注意哪几点?\,
    \begin{solution}
        \begin{enumerate}[(1)]
            \item 这一过程需要确定临界条件,\,包括控制棒的临界棒位以及临界硼浓度.\,
            \item 在启动过程中,\,需要在堆芯内装载中子源将核反应堆在启动和趋近临界过程中很低的中子注量率水平放大到中子计数器可监控的水平,\,使得整个核反应堆趋近临界的过程处于监督之下,\,避免核反应堆启动“盲区”.\,
        \end{enumerate}
    \end{solution}
\end{exercise}

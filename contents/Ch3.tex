\chapter{核反应堆中子学过程}
\section*{习题}

\begin{exercise}
    名词解释:\,有效增殖系数,\,慢化过程,\,扩散过程,\,中子流密度,\,中子慢化长度,\,中子扩散长度,\,中子徙动长度,\,中子慢化时间,\,中子扩散时间,\,中子平均寿命.\,
    \begin{solution}
        \begin{enumerate}[(1)]
            \item 有效增殖系数:\,一个系统内新生一代的中子数与产生它们的直属上一代中子数之比,\,即
            \begin{equation*}
                \keff = \frac{\text{新生一代中子数}}{\text{直属上一代中子数}} = \frac{\text{系统内中子产生率}}{\text{系统内中子总消失(吸收+泄漏)率}}
            \end{equation*}
            \item 慢化过程:\,中子能量不断减少,\,直至变成热中子的过程.\,
            \item 扩散过程:\,热中子在介质中位置不断变化的过程.\,
            \item 中子流密度:\,沿着某一方向穿过单位面积的中子数,\,用$\vec{J}$表示.\,
            \item 中子慢化长度:\,无限均匀介质中,\,快中子从点源出发至慢化成热中子所穿行的直线距离方均值的$\frac{1}{6}$. 
            \item 中子扩散长度:\,无限均匀介质中,\,热中子从产生至被吸收时所穿行的直线距离方均值的$\frac{1}{6}$. 
            \item 中子徙动长度:\,无限均匀介质中,\,快中子从产生到被吸收时所穿行的直线距离方均值的$\frac{1}{6}$. 
            \item 中子慢化时间:\,裂变中子从裂变能慢化到热中子分界能$E_{\symrm{th}}$所需的平均时间,\,用$t_{\symrm{s}}$表示.\,
            \item 中子扩散时间:\,热中子扩散至被吸收所需的平均时间,\,用$t_{\symrm{d}}$表示.\,
            \item 中子平均寿命:\,快中子自裂变慢化到热中子,\,再扩散到被吸收所需的平均总时间,\,用$l$表示,\,$l = t_{\symrm{s}} + t_{\symrm{d}}$.\,
        \end{enumerate}
    \end{solution}
\end{exercise}

\begin{exercise}
    某裂变堆,\,快中子增殖因数1.05,\,逃脱共振吸收概率0.9,\,慢化不泄漏概率0.952,\,扩散不泄漏概率0.94,\,有效裂变中子数1.335,\,热中子利用系数0.882,\,试计算其无限介质增殖因数和有效增殖因数.\,
    \begin{solution}
        由题意,\,$\varepsilon = 1.05,\,p=0.9,\,P_{\symrm{s}}=0.952,\,P_{\symrm{d}}=0.94,\,\eta=1.335,\,f=0.882$,\,则
        \begin{align*}
            &k_{\infty} = \varepsilon p f \eta = 1.05 \times 0.9 \times 0.882 \times 1.335 = 1.1127 \\
            &\keff = k_{\infty} P_{\symrm{s}} P_{\symrm{d}} = 1.1127 \times 0.952 \times 0.94 = 0.9957
        \end{align*}
    \end{solution}
\end{exercise}

\begin{exercise}
    某热中子核反应堆处于临界状态,\,每次裂变产生中子数$\nu = 2.43$,\,已知该堆的中子泄漏损失占总中子数的10\%,\,试求:\,
    \begin{enumerate}[(1)]
        \item 用于维持裂变反应的中子数占总中子数的百分比$R_{\symrm{f}}$;
        \item 除裂变以外,\,被吸收的中子数占总中子数的百分比$R_{\symrm{a}}$; 
    \end{enumerate}
    \begin{solution}
        由题意,\,$\keff = 1$,\,系统内中子产生=系统内中子(吸收+泄漏),\,而中子泄漏损失数为$0.1\nu$,\,则被吸收的中子数$0.9\nu$.
        \begin{enumerate}[(1)]
            \item 被吸收的中子中,\,用于维持裂变反应的中子数为$\dfrac{\sigma_{\symrm{f}}}{\sigma_{\symrm{a}}} \times 0.9\nu = \dfrac{0.9\nu}{1+\alpha}$(取$\alpha = 0.17$),\,于是
            \begin{equation*}
                R_{\symrm{f}} = \frac{\frac{0.9\nu}{1+\alpha}}{\nu} = \frac{0.9}{1+\alpha} = \frac{0.9}{1+0.17} = 0.7692
            \end{equation*}
            \item 除裂变以外,\,被吸收的中子数,\,即发生辐射俘获的中子数为$\dfrac{\sigma_{\gamma}}{\sigma_{\symrm{a}}} \times 0.9\nu = \dfrac{0.9 \alpha \nu}{1+\alpha}$,\,于是
            \begin{equation*}
                R_{\symrm{a}} = \frac{\frac{0.9 \alpha \nu}{1+\alpha}}{\nu} = \frac{0.9\alpha}{1+\alpha} = \frac{0.9\times 0.17}{1+0.17} = 0.1308
            \end{equation*}
        \end{enumerate}
    \end{solution}
\end{exercise}

\begin{exercise}
    核反应堆刚好临界时,\,$\keff = 1,\,\Delta k/k = 0$.
\end{exercise}

\begin{exercise}
    反应性的单位有哪些?
    \begin{solution}
        \begin{enumerate}[(1)]
            \item $1\,\symrm{pcm} = 10^{-5}$;
            \item $1\,\symrm{mk} = 10^{-3}$;
            \item $1\,\$ = 1\,\beta = 0.0065$(不是固定值).
        \end{enumerate}
    \end{solution}
\end{exercise}

\begin{exercise}
    操作员从堆中将控制棒提出,\,使得核反应堆的有效增殖因子$\keff$从0.998变为1.002,\,此核反应堆处于\xparen
    \begin{xchoices}[showanswer=true]
        \item 瞬发临界
        \item* 超临界
        \item 刚好临界
        \item 次临界
    \end{xchoices}
    \vspace{1em}
    \noindent {\color{third}{【注】}} {\kaishu 瞬发临界是指$\rho = \beta$,\,进一步,\,瞬发超临界是指$\rho > \beta$.}
\end{exercise}

\begin{exercise}
    在一个运行着的核反应堆堆芯中,\,一个热中子即将与一个铀-238核相互作用.\,以下哪一种情形最有可能发生,\,并且将怎样影响堆芯的$\keff$?\xparen
    \begin{xchoices}[showanswer=true]
        \item* 该中子将被散射,\,使$\keff$不变
        \item 该中子将被吸收,\,铀-238核将裂变,\,使$\keff$减小
        \item 该中子将被吸收,\,铀-238核将裂变,\,使$\keff$增大
        \item 该中子将被吸收,\,铀-238核将衰变,\,生成钚-239,\,使$\keff$增大
    \end{xchoices}
    \vspace{1em}
    \begin{solution}
        取热中子能量$E_{\symrm{n}}=0.0253\,\symrm{eV}$,\,查附录3,\,得${}^{238}\symrm{U}$截面数据:\,$\sigma_{\symrm{s}} = 9.299832\,\symrm{b},\,\sigma_{\gamma} = 2.682808\,\symrm{b},\,\sigma_{\symrm{f}} = 1.679563\times 10^{-5}\,\symrm{b}$,\,显然有$\sigma_{\symrm{s}} > \sigma_{\gamma} > \sigma_{\symrm{f}}$,\,即最有可能发生散射,\,则此时该热中子既不引发新的裂变产生新的中子,\,又不被吸收,\,也没有发生泄漏,\,故$\keff$不变,\,选A.\,
    \end{solution}
\end{exercise}

\begin{exercise}
    为使铀的有效裂变中子数$\eta=1.7$,\,试采用$E_{\symrm{n}}=0.0253\,\symrm{eV}$时的截面估计铀中${}^{235}\symrm{U}$的质量富集度.\,
    \begin{solution}
        对于$E_{\symrm{n}}=0.0253\,\symrm{eV}$的热中子,\,只有${}^{235}\symrm{U}$发生裂变,\,则有效裂变中子数
        \begin{equation*}
            \eta = \frac{\nu \vSigma_{\symrm{f5}}}{\vSigma_{\symrm{a5}}+\vSigma_{\symrm{a8}}} = \frac{\nu N_5 \sigma_{\symrm{f5}}}{N_5 \sigma_{\symrm{a5}} + N_8 \sigma_{\symrm{a8}}}
        \end{equation*}
        可以得到
        \begin{equation*}
            \frac{N_8}{N_5} = \frac{\nu \sigma_{\symrm{f5}} - \eta \sigma_{\symrm{a5}}}{\eta \sigma_{\symrm{a8}}}
        \end{equation*}
        进一步
        \begin{equation*}
            c_5 = \frac{N_5}{N_5+N_8} = \frac{1}{1+\frac{N_8}{N_5}} = \frac{\eta \sigma_{\symrm{a8}}}{\nu \sigma_{\symrm{f5}} - \eta \sigma_{\symrm{a5}} + \eta \sigma_{\symrm{a8}}}
        \end{equation*}
        代入$\eta = 1.7$,\,查附录3,\,得$\sigma_{\symrm{a8}}=2.6828\,\symrm{b},\,\sigma_{\symrm{a5}}=683.5565\,\symrm{b},\,\sigma_{\symrm{f5}} = 584.8925\,\symrm{b}$,\,取$\nu = 2.43$,\,得$c_5 = 0.0172885$.
        故铀中${}^{235}\symrm{U}$的质量富集度
        \begin{equation*}
            \varepsilon = \frac{235c_5}{235c_5+238(1-c_5)} = 0.01707 = 1.707\%
        \end{equation*}
    \end{solution}
\end{exercise}

\begin{exercise}
    某核反应堆堆芯内的平均宏观裂变截面为$5\,\symrm{m^{-1}}$,\,平均功率密度为$20\,\symrm{MW/m^3}$;\,假设每次裂变释放出的能量为200\,MeV,\,试求堆芯内的平均中子注量率.\,
    \begin{solution}
        由题意,\,$\vSigma_{\symrm{f}} = 5\,\symrm{m^{-1}},\,P_V = 20\,\symrm{MW/m^3} = 2\times 10^7\,\symrm{J/(m^3 \cdot s)},\,E_0 = 200\,\symrm{MeV} = 3.2\times 10^{-11}\,\symrm{J}$.\,由$P_V = \vSigma_{\symrm{f}} \overline{\phi} E_0$,\,得
        \begin{equation*}
            \overline{\phi} = \frac{P_V}{\vSigma_{\symrm{f}} E_0} = \frac{2\times 10^7}{5\times 3.2\times 10^{-11}}\,\symrm{m^{-2}\cdot s^{-1}} = 1.25 \times 10^{17}\,\symrm{m^{-2}\cdot s^{-1}}
        \end{equation*}
    \end{solution}
\end{exercise}

\begin{exercise}
    H和O在1000\,eV到1\,eV能量范围内的散射截面近似为常数,\,分别为20\,b和38\,b.\,计算$\symrm{H_2O}$的平均对数能降增量以及中子在$\symrm{H_2O}$中从1000\,eV慢化到1\,eV所需的平均碰撞次数.\,
    \begin{solution}
        平均对数能降增量
        \begin{equation*}
            \xi = 1 - \frac{(A-1)^2}{2A}\loge\left(\frac{A+1}{A-1}\right)
        \end{equation*}
        于是有$\xi_{\symrm{H}}=1,\,\xi_{\symrm{O}}=0.1199$.
        轻水的慢化能力来自氢核,\,氧核两方面的贡献,\,即
        \begin{align*}
            &\xi_{\symrm{H_2O}}\vSigma_{s,\,H_2O} = \xi_{\symrm{H}}\vSigma_{s,\,H} + \xi_{\symrm{O}}\vSigma_{s,\,O} \\
            \Rightarrow & \xi_{\symrm{H_2O}}(N_{\symrm{H}}\sigma_{\symrm{H}} + N_{\symrm{O}}\sigma_{\symrm{O}}) = \xi_{\symrm{H}}N_{\symrm{H}}\sigma_{\symrm{H}} + \xi_{\symrm{O}}N_{\symrm{O}}\sigma_{\symrm{O}}
        \end{align*}
        $N_{\symrm{H}}=2N_{\symrm{O}}$,\,两边同除以$N_{\symrm{O}}$,\,得
        \begin{equation*}
            \xi_{\symrm{H_2O}}(2\sigma_{\symrm{H}} + \sigma_{\symrm{O}}) = 2\xi_{\symrm{H}}\sigma_{\symrm{H}} + \xi_{\symrm{O}}\sigma_{\symrm{O}}
        \end{equation*}
        即
        \begin{equation*}
            \xi_{\symrm{H_2O}} = \frac{2\xi_{\symrm{H}}\sigma_{\symrm{H}} + \xi_{\symrm{O}}\sigma_{\symrm{O}}}{2\sigma_{\symrm{H}} + \sigma_{\symrm{O}}} = \frac{2\times 1\times 20 + 0.1199\times 38}{2\times 20 + 38} = 0.5712
        \end{equation*}
        平均碰撞次数
        \begin{equation*}
            N_{\symrm{c,\,H_2O}} = \frac{\loge \frac{E_1}{E_2}}{\xi_{\symrm{H_2O}}} = \frac{\loge \frac{1000}{1}}{0.5712} = 12.09
        \end{equation*}
    \end{solution}
\end{exercise}
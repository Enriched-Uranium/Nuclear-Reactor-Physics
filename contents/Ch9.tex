\chapter{堆芯核燃料管理}
\section*{习题}

\begin{exercise}
    名词解释:\,循环长度,\,换料周期,\,多循环燃料管理,\,单循环燃料管理,\,初始循环,\,平衡循环,\,过渡循环,\,线性反应性模型,\,增殖比,\,次锕系核素,\,长寿命裂变产物,\,核废料嬗变,\,闭式燃料循环.\,
    \begin{solution}
        \begin{enumerate}[(1)]
            \item 循环长度:\,一次装料后,\,核反应堆能满功率运行的时间;\,
            \item 换料周期:\,核反应堆两次停堆换料之间的时间间隔;\,
            \item 多循环燃料管理:\,燃料组件在堆芯内的空间影响仅以“批”的特性简单考虑,\,而在时间上可以考虑多个循环;\,
            \item 单循环燃料管理:\,详细考虑燃料组件和控制毒物在堆芯内的空间布置,\,可以不考虑各循环之间的相互影响;\,
            \item 初始循环:\,核反应堆首次启动运行的第一个循环,\,堆芯全部由新燃料组成;\,
            \item 平衡循环:\,每个循环的性能参数都保持相同,\,运行循环进入到一个平衡状态;\,
            \item 过渡循环:\,从第2循环开始一直延续到平衡循环为止的各个循环;\,
            \item 线性反应性模型:\,对于典型轻水堆燃料组件,\,其反应性$\rho$可近似为燃料燃耗深度的线性递减函数,\,即
            \begin{equation*}
                \rho_i(B) = \rho_{0,i} - \alpha_i B_i
            \end{equation*}
            \item 增殖比:\,转换比(CR)>1时的转换比称为增殖比,\,用BR表示;\,
            \item 次锕系核素:\,除去U和Pu以外的其他锕系元素,\,简称MA;\,
            \item 长寿命裂变产物:\,半衰期特别长,\,已远远超出人类管理范围的裂变产物;\,
            \item 核废料嬗变:\,通过中子俘获反应,\,把长寿命高放同位素变成短寿命或稳定同位素;\,
            \item 闭式燃料循环:\,核燃料从地质勘探、采矿、铀浓缩、燃料组件制造到核反应堆内燃烧、后处理、地质贮存等过程,\,形成一个封闭的核燃料循环.\,
        \end{enumerate}
    \end{solution}
\end{exercise}

\begin{exercise}
    设一座核反应堆由$n = 3$批料组成,\,假设反应性随燃耗线性减少,\,且其斜率与初始富集度无关,\,各批料以相同功率密度运行,\,初始循环的循环燃耗与以后各循环的相等,\,从第2循环开始新料的富集度就取成平衡循环的换料富集度,\,请确定各批料的初始反应性.\,
    \begin{solution}
        设循环燃耗深度为$B_n^c$,\,每批料反应性变化斜率为$\alpha$,\,根据满功率运行循环寿期末堆芯反应性为0,\,根据线性反应性模型$\rho_i(B) = \rho_{0,i} - \alpha_i B_i$,\,得
        \begin{equation*}
            \rho_0 - \frac{1}{n}\sum_{i=1}^{n} i \alpha B_n^c = 0
        \end{equation*}
        于是有循环燃耗为
        \begin{equation*}
            B_n^c = \frac{2\rho_0}{(n+1)\alpha}
        \end{equation*}
        进一步,\,卸料燃耗为
        \begin{equation*}
            B_n^d = n B_n^c = \frac{2n}{n+1}B_1^d = \frac{2n}{n+1}B_1^c = \frac{2n\rho_0}{(n+1)\alpha}
        \end{equation*}
        则初始反应性
        \begin{equation*}
            \rho_0 = \frac{n+1}{2n} \alpha B_n^d = \frac{n+1}{2} \alpha B_n^c = 2\alpha B_3^c
        \end{equation*}
    \end{solution}
\end{exercise}

\begin{exercise}
    试从我国国情出发论述我国发展快堆的必要性.\,
    \begin{solution}
        我国铀矿储量少,\,核废料处理技术发展尚不成熟.\,快堆在释放能量的时候同时能够核燃料增殖,\,还能将乏燃料中的长寿命废料变成短寿命废料.\,这是实现我国核能可持续发展的关键.\,
    \end{solution}
\end{exercise}
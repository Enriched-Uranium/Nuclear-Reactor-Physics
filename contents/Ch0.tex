\chapter*{写在前面}

《核反应堆物理》是西安交通大学核工程专业学生在大三下学期的必修课,旨在通过模拟核反应堆内中子与原子核的相互作用过程,为堆芯核设计、堆芯燃料管理、核反应堆运行、核反应堆启动试验、核反应堆安全分析等提供中子学基础数据。

我们希望通过这份资料给出课后习题的解答,帮助学习这门课的同学更好地完成作业以及高效的期末复习备考。

这份解答主要由\;{\kaishu 一小块浓缩铀}\;撰稿及排版,由\;{\kaishu 刘Sir}\;审核。在2022-2023学年第二学期已经有部分2020级核工程专业的同学在完成作业以及复习备考时因此资料而获益,反响较好。因为我们水平有限,错漏之处在所难免,如果您发现有计算错误、笔误或其他需要改进之处,可以前往GitHub项目地址提供Issue或直接联系微信:

\begin{itemize}
    \item {\faGithub}:\,\href{https://github.com/Enriched-Uranium/Nuclear-Reactor-Physics}{https://github.com/Enriched-Uranium/Nuclear-Reactor-Physics}
    \item {\faWeixin}:\,XJTU-NEer
\end{itemize}

整份文档格式由\LaTeX{}排版,模板来自\href{https://zhuanlan.zhihu.com/p/601085820}{\faZhihu}作者Jiann,并在其基础上做了些许修改,图片使用Adobe illustrator和AxGlyph绘制。

酒红色文本均已嵌入超链接,点击即可跳转。

\begin{flushright}
    {\kaishu 一小块浓缩铀 \& 刘Sir} \\
    \zhtoday
\end{flushright}
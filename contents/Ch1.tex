\chapter{绪论}

\section*{习题}

\begin{exercise}
    什么是核反应堆?\,按照核反应堆发生的机理可分为哪几类?\,按照堆内的中子能谱可分为哪几类?\,按照冷却剂和慢化剂的种类可分为哪几类?\,
    \begin{solution}
        \begin{enumerate}[(1)]
            \item 核反应堆是指能以可控方式实现自持的链式裂变反应或核聚变反应的装置.
            \item 按照核反应堆发生的机理,核反应堆可分为裂变核反应堆、聚变核反应堆和聚变-裂变混合堆.
            \item 按照堆内的中子能谱,核反应堆可分为热中子堆和快中子堆.
            \item 按照冷却剂和慢化剂的种类,核反应堆可分为轻水堆(压水堆和沸水堆)、重水堆、气冷堆、液态金属堆等.
        \end{enumerate}
    \end{solution}
\end{exercise}

\begin{exercise}
    核反应堆物理分析的主要目标是什么?\,
    \begin{solution}
        核反应堆物理分析的主要目标是通过模拟核反应堆内中子与原子核的相互作用过程,为堆芯核设计、堆芯燃料管理、核反应堆运行、核反应堆启动试验、核反应堆安全分析等提供中子学基础数据.
    \end{solution}
\end{exercise}
\chapter{中子扩散理论与计算}
\section*{习题}

\begin{exercise}
    试总结中子扩散理论的适用范围。
    \begin{solution}
        \begin{enumerate}[(1)]
            \item 介质无限均匀;
            \item 在L系内散射各向同性;
            \item 介质弱吸收;
            \item 中子注量率分布变化缓慢。
        \end{enumerate}
    \end{solution}
\end{exercise}

\begin{exercise}
    有两束方向相反的平行热中子束射到${}^{235}\symrm{U}$薄片上,设其上某点自左面入射的中子束强度为$10^{12}\,\symrm{cm^{-2}\cdot s^{-1}}$,自右面入射的中子束强度$2 \times 10^{12}\,\symrm{cm^{-2}\cdot s^{-1}}$。计算:
    \begin{enumerate}[(1)]
        \item 该点的中子注量率;
        \item 该点的中子流密度;
        \item 设$\vSigma_{\symrm{a}} = 19.2 \times 10^2\,\symrm{m^{-1}}$,求该点的吸收率。
    \end{enumerate}
    \begin{solution}
        \begin{figure}[H]
            \centering
            \includegraphics[scale=1.5]{figures/fig4.2.png}
        \end{figure}
        如图所示,取水平向右为正方向,则
        \begin{enumerate}[(1)]
            \item $\phi = I^{+} + I^{-} = 3\times 10^{12}\,\symrm{cm^{-2}\cdot s^{-1}}$;
            \item $J = I^{+} - I^{-} = -10^{12}\,\symrm{cm^{-2}\cdot s^{-1}}$,\,负号表示其方向与正方向相反;
            \item $R_{\symrm{a}} = \vSigma_{\symrm{a}}\phi = 19.2\times 3\times 10^{12}\,\symrm{cm^{-3}\cdot s^{-1}} = 5.76\times 10^{13}\,\symrm{cm^{-3}\cdot s^{-1}}$.
        \end{enumerate}
    \end{solution}
\end{exercise}

\begin{exercise}
    在某球形裸堆($R=0.5\,\symrm{m}$)内中子注量率分布为:
    \begin{equation*}
        \phi(r) = \frac{5\times 10^{13}\,\symrm{cm^{-1}\cdot s^{-1}}}{r} \nsin \left(\frac{\pi r}{R}\right)\,\symrm{cm^{-2}\cdot s^{-1}}
    \end{equation*}
    试求:
    \begin{enumerate}[(1)]
        \item $\phi(0)$;
        \item $J(r)$的表达式,设$D = 0.8\times 10^{-2}\,\symrm{m}$;
        \item 每秒从堆表面泄漏的总中子数(假设外推距离很小可忽略不计)。
    \end{enumerate}
    \begin{solution}
        \begin{enumerate}[(1)]
            \item \begin{equation*}
                \phi = \lim_{r\to 0} \frac{5\times 10^{13}}{r} \nsin \left(\frac{\pi r}{R}\right) = \lim_{r\to 0} \frac{5\times 10^{13}}{r} \cdot \frac{\pi r}{R} = 5\times 10^{13} \times \frac{\pi}{50}\,\symrm{cm^{-2}\cdot s^{-1}} = \pi \times 10^{12}\,\symrm{cm^{-2}\cdot s^{-1}}
            \end{equation*}
            \item $D = 0.8\times 10^{-2}\,\symrm{m} = 0.8\,\symrm{cm}$
            \begin{equation*}
                J(r) = -D\nabla \phi(r) = -D \dv{\phi(r)}{r} = 4\times 10^{13} \left[\frac{1}{r^2}\nsin \left(\frac{\pi r}{50}\right) - \frac{\pi}{50 r} \ncos\left(\frac{\pi r}{50}\right)\right]\,\symrm{cm^{-2}\cdot s^{-1}}
            \end{equation*}
            \item \begin{equation*}
                L = J(R)\cdot 4\pi R^2 = 4\times 10^{13} \left[\frac{1}{50^2}\nsin \left(\frac{50 \pi}{50}\right) - \frac{\pi}{50 \times 50} \ncos\left(\frac{50\pi}{50}\right)\right]\times 4\pi\times 50^2 \,\symrm{s^{-1}} = 1.579\times 10^{15}\,\symrm{s^{-1}}
            \end{equation*}
        \end{enumerate}
    \end{solution}
\end{exercise}

\begin{exercise}
    无限纯吸收介质(中子扩散系数为$D$,中子平均自由程为$\lambda$)内,在坐标$(-a,\,0,\,0)$和$(a,\,0,\,0)$处分别有两个源强为$S\,\symrm{s^{-1}}$的点源,试求坐标原点$P_1$和坐标$(0,\,a,\,0)\,P_2$处的中子注量率和中子流密度。
    \begin{solution}
        取$xy$平面,如图所示
        \begin{figure}[H]
            \centering
            \includegraphics[scale=1.5]{figures/fig4.4.png}
        \end{figure}
        以$S_1$为原点建立一维球坐标系$S_1-r$,则单能中子稳态扩散方程
        \begin{equation*}
            \nabla^2\phi(r)-\frac{\phi(r)}{L^2} = 0,\,r>0
        \end{equation*}
        其中,$r^2 = (x+a)^2+y^2$,$x\neq -a$且$y\neq 0$。

        通解为
        \begin{equation*}
            \phi(r) = A\frac{\symrm{e}^{-r/L}}{r} + C\frac{\symrm{e}^{r/L}}{r}
        \end{equation*}
        边界条件
        \begin{enumerate}[(a)]
            \item $\phi(r)$为有限正值,于是$C=0$;
            \item $\lim\limits_{r\to 0} J(r)\cdot 4\pi r^2 = S$
        \end{enumerate}

        解得
        \begin{align*}
            &J(r) = \frac{S}{4\pi r}\left(\frac{1}{L}+\frac{1}{r}\right)\symrm{e}^{-r/L} = \frac{S}{4\pi r}\left(\frac{1}{L}+\frac{1}{r}\right)\symrm{e}^{-r/\sqrt{D\lambda}} \\
            &\phi(r) = \frac{S\symrm{e}^{-r/L}}{4\pi D r} = \frac{S\symrm{e}^{-r/\sqrt{D\lambda}}}{4\pi D r}
        \end{align*}
        $S_2$与$S_1$源强相等且对称,则
        \begin{enumerate}[(1)]
            \item $P_1$点
            \begin{align*}
                &J_{P_1} = 0 \\
                &\phi_{P_1} = 2\phi(a) = \frac{S\symrm{e}^{-a/\sqrt{D\lambda}}}{2\pi D a}
            \end{align*}
            \item $P_2$点
            \begin{align*}
                &J_{P_2} = \sqrt{2}J(\sqrt{2}a) = \frac{S}{4\pi a}\left(\frac{1}{L}+\frac{1}{\sqrt{2}a}\right)\symrm{e}^{-\sqrt{2}a/\sqrt{D\lambda}} \\
                &\phi_{P_2} = 2\phi(\sqrt{2}a) = \frac{\sqrt{2}S\symrm{e}^{-\sqrt{2}a/\sqrt{D\lambda}}}{4\pi D a}
            \end{align*}
        \end{enumerate}
    \end{solution}
\end{exercise}

\begin{exercise}
    试求边长为$a,\,b,\,c$(包括外推距离)的长方体裸堆的几何曲率和中子注量率分布。设有一边长$a = b = 0.5\,\symrm{m},\,c = 0.6\,\symrm{m}$(包括外推距离)的长方体裸堆,$L = 0.0434\,\symrm{m},\,\tau = 6\,\symrm{cm^2}$。
    \begin{enumerate}[(1)]
        \item 求达到临界时所必需的$k_{\infty}$;
        \item 如果功率为$5000\,\symrm{kW},\,\vSigma_{\symrm{f}} = 4.01\,\symrm{m^{-1}}$,假设每次裂片释放出的能量为$200\,\symrm{MeV}$,求中子注量率分布。
    \end{enumerate}
    \begin{solution}
        以长方体几何中心为坐标原点建立空间直角坐标系$Oxyz$,如图所示
        \begin{figure}[H]
            \centering
            \includegraphics[scale=1]{figures/fig4.5.png}
        \end{figure}
        单能稳态中子扩散方程
        \begin{equation*}
            D\left(\pddv{\phi}{x}+\pddv{\phi}{y}+\pddv{\phi}{z}\right) - \vSigma_{\symrm{a}}\phi + k_{\infty}\vSigma_{\symrm{a}}\phi = 0
        \end{equation*}

        边界条件
        \begin{equation*}
            \phi\left(\pm \frac{a}{2},\,y,\,z\right) = \phi\left(x,\,\pm \frac{b}{2},\,z\right) = \phi\left(x,\,y,\,\pm \frac{c}{2}\right) = 0
        \end{equation*}

        分离变量$\phi(x,\,y,\,z) = \varphi_x(x)\varphi_y(y)\varphi_y(y)$,代入扩散方程,得
        \begin{equation*}
            \frac{\nabla^2\varphi_x(x)}{\varphi_x(x)} + \frac{\nabla^2\varphi_y(y)}{\varphi_y(y)} + \frac{\nabla^2\varphi_z(z)}{\varphi_z(z)} = -\frac{k_{\infty}-1}{L^2}
        \end{equation*}

        令
        \begin{equation*}
            \frac{\nabla^2\varphi_x(x)}{\varphi_x(x)} = -B_x^2 \Rightarrow \nabla^2\varphi_x(x) + B_x^2\varphi_x(x) = 0
        \end{equation*}
        通解为$\varphi_x(x) = A_{nx}\ncos B_{nx}x + C_{nx}\nsin B_{nx}x$。

        \begin{enumerate}[(a)]
            \item 由通量对称分布,得$C_{nx}=0$;
            \item 由边界条件,得$\varphi_x(a/2) = A_{nx}\ncos(aB_{nx}/2) = 0 \Rightarrow B_{nx} = \frac{(2n-1)\pi}{a}\,n=1,\,2,\,3,\,\cdots$.
        \end{enumerate}

        稳态时,取$B_{1x} = \frac{\pi}{a}$,同理,$B_{1y} = \frac{\pi}{b},\,B_{1z} = \frac{\pi}{c}$,\,故几何曲率和中子注量率分布为
        \begin{align*}
            &B_{\symrm{g}}^2 = \left(\frac{\pi}{a}\right)^2 + \left(\frac{\pi}{b}\right)^2 + \left(\frac{\pi}{c}\right)^2 \\
            &\phi = A'_n \ncos\left(\frac{\pi}{a}x\right) \ncos\left(\frac{\pi}{b}y\right) \ncos\left(\frac{\pi}{c}z\right)
        \end{align*}

        \begin{enumerate}[(1)]
            \item 由题意,得几何曲率
            \begin{equation*}
                B_{\symrm{g}}^2 = \left(\frac{\pi}{0.5}\right)^2 + \left(\frac{\pi}{0.5}\right)^2 + \left(\frac{\pi}{0.6}\right)^2\,\symrm{m^{-2}} = 106.4\,\symrm{m^{-2}}
            \end{equation*}
            临界时,满足
            \begin{equation*}
                \frac{k_{\infty} - 1}{M^2} = B_{\symrm{g}}^2 \Rightarrow k_{\infty} = B_{\symrm{g}}^2(L^2+\tau) + 1 = 106.4 \times (0.0434^2 + 0.0006) + 1 = 1.264
            \end{equation*}
            \item 由题意,有
            \begin{equation*}
                P = E_{\symrm{f}}\int_V \vSigma_{\symrm{f}} \phi \dd{V} = E_{\symrm{f}} \vSigma_{\symrm{f}} A'_n \int_{-a/2}^{a/2} \ncos\left(\frac{\pi}{a}x\right) \dd{x} \int_{-b/2}^{b/2} \ncos\left(\frac{\pi}{b}y\right) \dd{y} \int_{-c/2}^{c/2} \ncos\left(\frac{\pi}{c}z\right) \dd{z} = E_{\symrm{f}} \vSigma_{\symrm{f}} A'_n abc \left(\frac{2}{\pi}\right)^3
            \end{equation*}
            于是
            \begin{equation*}
                A'_n = \frac{P(\pi/2)^3}{E_{\symrm{f}} \vSigma_{\symrm{f}} abc} = \frac{5\times 10^6 \times (\pi/2)^3}{200 \times 10^6 \times 1.6 \times 10^{-19} \times 4.01 \times 0.5 \times 0.5 \times 0.6}\,\symrm{m^{-2}\cdot s^{-1}} = 1.007 \times 10^{18}\,\symrm{m^{-2}\cdot s^{-1}}
            \end{equation*}
            故中子注量率分布
            \begin{equation*}
                \phi(x,\,y,\,z) = 1.007 \times 10^{18} \ncos\left(2\pi x\right) \ncos\left(2\pi y\right) \ncos\left(\frac{5\pi}{3}z\right)\,\symrm{m^{-2}\cdot s^{-1}}
            \end{equation*}
        \end{enumerate}
    \end{solution}
\end{exercise}

\begin{exercise}
    设一座重水-铀核反应堆堆芯的$k_{\infty} = 1.28,\,L^2 = 1.8\times 10^{-2}\,\symrm{m^2},\,\tau = 1.20\times 10^{-2}\,\symrm{m^2}$。试按单群理论,修正单群理论的临界方程分别求出该堆芯的材料曲率和达到临界时总的中子不泄漏概率。
    \begin{solution}
        \begin{enumerate}[(1)]
            \item 按单群理论
            \begin{align*}
                &B_{\symrm{m}}^2 = \frac{k_{\infty} - 1}{L^2} = \frac{1.28 - 1}{1.8\times 10^{-2}}\,\symrm{m^{-2}} = 15.56\,\symrm{m^{-2}} \\
                &P_L = \frac{1}{1 + L^2 B_{\symrm{g}}^2} = \frac{1}{1 + L^2 B_{\symrm{m}}^2} = \frac{1}{1 + 1.8\times 10^{-2}\times 15.56} = 0.7812
            \end{align*}
            \item 按修正单群理论
            \begin{align*}
                &B_{\symrm{m}}^2 = \frac{k_{\infty} - 1}{M^2} = \frac{k_{\infty} - 1}{L^2 + \tau} = \frac{1.28 - 1}{1.8\times 10^{-2} + 1.2\times 10^{-2}}\,\symrm{m^{-2}} = 9.333\,\symrm{m^{-2}} \\
                &P_L = \frac{1}{1 + M^2 B_{\symrm{g}}^2} = \frac{1}{1 + (L^2 + \tau) B_{\symrm{m}}^2} = \frac{1}{1 + (1.8\times 10^{-2} + 1.2\times 10^{-2})\times 9.333} = 0.7813
            \end{align*}
        \end{enumerate}
    \end{solution}
\end{exercise}

\begin{exercise}
    设有圆柱形铀一水栅格装置,$R = 0.50\,\symrm{m}$,水位高度$H = 1.0\,\symrm{m}$,设栅格参数为:$k_{\infty} = 1.19,\,L^2 = 6.6\times 10^{-4}\,\symrm{m^2},\,\tau = 0.50\times 10^{-2}\,\symrm{m^2}$。
    \begin{enumerate}[(1)]
        \item 试求该装置的有效增殖因数$\keff$;
        \item 当该装置恰好达到临界时,水位高度$H$等于多少?
        \item 设某压水堆以该铀-水栅格作为芯部,堆芯的尺寸为$R = 1.66\,\symrm{m},\,H = 3.5\,\symrm{m}$,若反射层节省估算为$\delta_{\symrm{r}} = 0.07\,\symrm{m},\,\delta_{\symrm{H}} = 0.1\,\symrm{m}$,试求核反应堆的初始反应性$\rho_0$。
    \end{enumerate}
    \begin{solution}
        假设$R=0.5\,\symrm{m},\,H=1.0\,\symrm{m}$已包含外推距离。
        \begin{enumerate}[(1)]
            \item 几何曲率
            \begin{equation*}
                B_{\symrm{g}}^2 = \left(\frac{2.405}{R}\right)^2 + \left(\frac{\pi}{H}\right)^2 = \left(\frac{2.405}{0.5}\right)^2 + \left(\frac{\pi}{1}\right)^2\,\symrm{m^{-2}} = 33.01\,\symrm{m^{-2}}
            \end{equation*}
            有效增殖因数
            \begin{equation*}
                \keff = \frac{k_{\infty}}{1 + M^2 B_{\symrm{g}}^2} = \frac{k_{\infty}}{1 + (L^2 + \tau) B_{\symrm{g}}^2} = \frac{1.19}{1 + (6.6\times 10^{-4} + 0.5\times 10^{-2}) \times 33.01} = 1.003
            \end{equation*}
            \item 材料曲率
            \begin{equation*}
                B_{\symrm{m}}^2 = \frac{k_{\infty} - 1}{M^2} = \frac{k_{\infty} - 1}{L^2 + \tau} = \frac{1.19 - 1}{6.6\times 10^{-4} + 0.5\times 10^{-2}}\,\symrm{m^{-2}} = 33.57\,\symrm{m^{-2}}
            \end{equation*}
            临界时,有
            \begin{equation*}
                B_{\symrm{g}}^2 = \left(\frac{2.405}{R}\right)^2 + \left(\frac{\pi}{H}\right)^2 = B_{\symrm{m}}^2 = 33.57\,\symrm{m^{-2}}
            \end{equation*}
            
            由此解得$H = 0.9726\,\symrm{m}$.
            \item 等效裸堆尺寸
            \begin{equation*}
                \begin{cases}
                    R_{\symrm{eff}} = R + \delta_{\symrm{r}} = 1.66 + 0.07\,\symrm{m} = 1.73\,\symrm{m} \\
                    H_{\symrm{eff}} = H + 2\delta_{\symrm{H}} = 3.5 + 2 \times 0.1 = 3.7\,\symrm{m}
                \end{cases}
            \end{equation*}
            几何曲率
            \begin{equation*}
                B_{\symrm{g}}^2 = \left(\frac{2.405}{R_{\symrm{eff}}}\right)^2 + \left(\frac{\pi}{H_{\symrm{eff}}}\right)^2 = \left(\frac{2.405}{1.73}\right)^2 + \left(\frac{\pi}{3.7}\right)^2\,\symrm{m^{-2}} = 2.654\,\symrm{m^{-2}}
            \end{equation*}
            有效增殖因数
            \begin{equation*}
                \keff = \frac{k_{\infty}}{1 + M^2 B_{\symrm{g}}^2} = \frac{k_{\infty}}{1 + (L^2 + \tau) B_{\symrm{g}}^2} = \frac{1.19}{1 + (6.6\times 10^{-4} + 0.5\times 10^{-2}) \times 2.654} = 1.172
            \end{equation*}
            初始反应性
            \begin{equation*}
                \rho_0 = \frac{\keff - 1}{\keff} = \frac{1.172 - 1}{1.172} = 0.1468
            \end{equation*}
        \end{enumerate}
    \end{solution}
\end{exercise}

\begin{exercise}
    一球壳形核反应堆,内半径为$R_1$,外半径为$R_2$(包含外推距离),如果球的内外均为真空,求证单群理论的临界条件为:
    \begin{equation*}
        \ntan BR_2 = \frac{\ntan BR_1 - BR_1}{BR_1\ntan BR_1 + 1}
    \end{equation*}
    \begin{solution}
        \begin{figure}[H]
            \centering
            \includegraphics[scale=1]{figures/fig4.8.png}
        \end{figure}
        以球心为坐标原点建立一维球坐标系,如图所示,则临界时单能中子稳态扩散方程(增殖)为
        \begin{equation*}
            \ddv{\phi(r)}{r} + \frac{2}{r} \dv{\phi(r)}{r} + \frac{k_{\infty} - 1}{L^2}\phi(r) = 0 \xrightarrow{B^2 = (k_{\infty} - 1)/L^2} \ddv{\phi(r)}{r} + \frac{2}{r} \dv{\phi(r)}{r} + B^2 \phi(r) = 0,\,r > 0
        \end{equation*}
        通解为
        \begin{equation*}
            \phi(r) = A\frac{\nsin Br}{r} + C\frac{\ncos Br}{r}
        \end{equation*}
        
        边界条件
        \begin{enumerate}[(1)]
            \item $\lim\limits_{r\to R_1} J = 0$
            \begin{equation*}
                \lim_{r\to R_1} J = \lim_{r\to R_1} -D\dv{\phi(r)}{r} = \lim_{r\to R_1} -D\left(AB\frac{\ncos Br}{r} - A\frac{\nsin Br}{r^2} - BC\frac{\nsin Br}{r} - C\frac{\ncos Br}{r^2}\right) = 0
            \end{equation*}
            而$D \neq 0$,立即推
            \begin{equation*}
                AB\frac{\ncos BR_1}{R_1} - A\frac{\nsin BR_1}{R_1^2} - BC\frac{\nsin BR_1}{R_1} - C\frac{\ncos BR_1}{R_1^2} = 0
            \end{equation*}
            解得
            \begin{equation*}
                C = A\frac{BR_1 - \ntan BR_1}{BR_1 \ntan BR_1 + 1}
            \end{equation*}
            \item $\phi(R_2) = 0$
            \begin{equation*}
                \phi(R_2) = A\frac{\nsin BR_2}{R_2} + C\frac{\ncos BR_2}{R_2} = 0 \Rightarrow C = -A\ntan BR_2
            \end{equation*}
        \end{enumerate}
        联立可得
        \begin{equation*}
            \ntan BR_2 = \frac{\ntan BR_1 - BR_1}{BR_1\ntan BR_1 + 1}
        \end{equation*}
    \end{solution}
\end{exercise}

\begin{exercise}
    一维无限平板几何下,坐标原点左侧和右侧分别为两种非增殖但含中子源的材料,两种材料的扩散系数相同、中子扩散长度的倒数分别为$\kappa_{\symrm{l}}^2$和$\kappa_{\symrm{r}}^2$,相应的中子源强分别为$\kappa_{\symrm{l}}^2 \vPhi_{\symrm{l}}$和$\kappa_{\symrm{r}}^2 \vPhi_{\symrm{r}}$。
    \begin{enumerate}[(1)]
        \item 试计算该一维无限平板内的中子注量率分布和原点处的中子注量率;
        \item 试画出区间$[-5/\kappa_{\symrm{l}},\,5/\kappa_{\symrm{r}}]$内在$\vPhi_{\symrm{l}}=0,\,\vPhi_{\symrm{r}}=1,\,\kappa_{\symrm{l}}=\kappa_{\symrm{r}}=\kappa$时的中子注量率分布;
        \item 试画出区间$[-5/\kappa_{\symrm{l}},\,5/\kappa_{\symrm{r}}]$内在$\vPhi_{\symrm{l}}=2,\,\vPhi_{\symrm{r}}=1,\,\kappa_{\symrm{l}}=3\kappa,\,\kappa_{\symrm{r}}=\kappa$时的中子注量率分布。
    \end{enumerate}
    \begin{solution}
        \begin{enumerate}[(1)]
            \item 如图所示,以$x>0$为例,有
            \begin{figure}[H]
                \centering
                \includegraphics[scale=1]{figures/fig4.9.png}
            \end{figure}
            \begin{equation*}
                \ddv{\phi_{\symrm{r}}(x)}{x} - \frac{\phi_{\symrm{r}}(x)}{L^2} = 0,\,x > 0
            \end{equation*}
            通解为
            \begin{equation*}
                \phi_{\symrm{r}}(x) = A_{\symrm{r}} \symrm{e}^{-x/L_r} + C_{\symrm{r}} \symrm{e}^{x/L_r}
            \end{equation*}
            边界条件
            \begin{enumerate}[(a)]
                \item $\phi_{\symrm{r}}(x)$为有限正值,于是$C_{\symrm{r}} = 0$;
                \item $\lim\limits_{x\to 0^{+}} J_{\symrm{r}}(x) = \kappa_{\symrm{r}}^2 \vPhi_{\symrm{r}} \Rightarrow A_{\symrm{r}} = \frac{\kappa_{\symrm{r}}^2 \vPhi_{\symrm{r}} L_{\symrm{r}}}{D} \xlongequal[\kappa_{\symrm{r}}^2 L_{\symrm{r}}^2 = 1]{\kappa_{\symrm{r}} L_{\symrm{r}} = 1} \frac{\kappa_{\symrm{r}} \vPhi_{\symrm{r}}}{D}$
            \end{enumerate}
            于是
            \begin{equation*}
                \phi_{\symrm{r}}(x) = \frac{\kappa_{\symrm{r}} \vPhi_{\symrm{r}}}{D} \symrm{e}^{-\kappa_{\symrm{r}}x},\,x\geqslant 0
            \end{equation*}
            同理
            \begin{equation*}
                \phi_{\symrm{l}}(x) = \frac{\kappa_{\symrm{l}} \vPhi_{\symrm{l}}}{D} \symrm{e}^{\kappa_{\symrm{l}}x},\,x < 0
            \end{equation*}
            即
            \begin{equation*}
                \phi(x) = \begin{cases}
                    \frac{\kappa_{\symrm{l}} \vPhi_{\symrm{l}}}{D} \symrm{e}^{\kappa_{\symrm{l}}x},\,x < 0 \\
                    \frac{\kappa_{\symrm{r}} \vPhi_{\symrm{r}}}{D} \symrm{e}^{-\kappa_{\symrm{r}}x},\,x > 0
                \end{cases}
            \end{equation*}
            原点处中子注量率不连续,有
            \begin{align*}
                &\phi(0^{-}) = \lim_{x \to 0} \phi_{\symrm{l}}(x) = \lim_{x \to 0} \frac{\kappa_{\symrm{l}} \vPhi_{\symrm{l}}}{D} \symrm{e}^{\kappa_{\symrm{l}}x} = \frac{\kappa_{\symrm{l}} \vPhi_{\symrm{l}}}{D} \\
                &\phi(0^{+}) = \lim_{x \to 0} \phi_{\symrm{r}}(x) = \lim_{x \to 0} \frac{\kappa_{\symrm{r}} \vPhi_{\symrm{r}}}{D} \symrm{e}^{-\kappa_{\symrm{r}}x} = \frac{\kappa_{\symrm{r}} \vPhi_{\symrm{r}}}{D}
            \end{align*}
            \item 由题意,此时
            \begin{equation*}
                \phi(x) = \begin{cases}
                    \phi_{\symrm{l}}(x) = 0,\,x < 0 \\
                    \phi_{\symrm{r}}(x) = \frac{\kappa}{D} \symrm{e}^{-\kappa x},\,x > 0
                \end{cases}
            \end{equation*}
            \begin{figure}[H]
                \centering
                \includegraphics[scale=1]{figures/fig4.9-2.png}
            \end{figure}
            \item 由题意,此时
            \begin{equation*}
                \phi(x) = \begin{cases}
                    \phi_{\symrm{l}}(x) = \frac{6\kappa}{D} \symrm{e}^{3\kappa x},\,x < 0 \\
                    \phi_{\symrm{r}}(x) = \frac{\kappa}{D} \symrm{e}^{-\kappa x},\,x > 0
                \end{cases}
            \end{equation*}
            \begin{figure}[H]
                \centering
                \includegraphics[scale=0.8]{figures/fig4.9-3.png}
            \end{figure}
        \end{enumerate}
    \end{solution}
\end{exercise}